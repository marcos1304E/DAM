\documentclass{article}
\usepackage{graphicx} % Required for inserting images
\usepackage{amsmath}
\usepackage{amssymb}

\title{01.03 EJERCICIOS DE IDENTIFICACIÓN DEL LENGUAJE DE MARCAS}
\author{FP Marcos Escamilla Ojeda}
\date{September 2024}

\begin{document}
\tableofcontents

\section{EJERCICIOS DE IDENTIFICACIÓN DEL LENGUAJE DE MARCAS}
\subsection{Ejercicio 1.}

$A\smallsetminus \left( B\cup C \right)=\left\{ -1,0,1,2,3 \right\}\smallsetminus\left( \left\{ 2 \right\}\cup \left\{ -\frac{1}{2},1 \right\} \right)$

\subsection{Ejercicio 2.}
$p \veebar\left( q \wedge\neg p \right)\equiv \left[ p\vee\left( q\wedge\neg p \right) \right]\wedge\neg \left[ p\wedge\left( q\wedge\neg p \right) \right]$            Eq. de la disyunción exclusiva.
\subsection{Ejercicio 3.}
     $\binom{20}{k}(x^{\frac{3}{2}})^{k} (-x^{-1})^{20-k}= \binom{20}{k}x^{\frac{3k}{2}}(-1)^{20-k}x^{k-20}$

\subsection{Ejercicio 4.}
$\frac{\left| 2x+5 \right|\left| x-4 \right|}{x-4}\le (x+1)^{2}\Longleftrightarrow \frac{-(2x+5)(-(x-4))}{x-4}\le (x+1)^{2}$
\subsection{Ejercicio 5.}
$\sqrt{\sqrt{x-1}}+\sqrt{\sqrt{x}+1} \lt  \sqrt{2\sqrt{x}}\Longleftrightarrow (\sqrt{\sqrt{x}-1}+\sqrt{\sqrt{x}+1})^{2} \lt (\sqrt{2\sqrt{x}})^{2}$
\subsection{Ejercicio 6.}

\begin{split}

$\sum_{k=1}^{n+1}k(k!)=\sum_{k=1}^{n}k(k!)+(n+1)((n+1)!)\\=(n+1)!-1+(n+1)((n+1)!)\\=(1+n+1)((n+1)!)-1\\=(n+2)((n+1)!)-1\\=(n+2)!-1.$
\end{split}

\end{document}
